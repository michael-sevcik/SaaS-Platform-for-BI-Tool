\chapter{Úvod}\label{chap:intro}
% \addcontentsline{toc}{chapter}{Úvod}


Pro jakoukoliv firmu nabízející vzájemně provázané služby je důležité mít v~rámci svého ekosystému takový produkt, který usnadní potenciálním zákazníkům cestu k~využití dalších produktů.
V~případě zadavatelské firmy, kterou pro jednoduchost budeme nazývat \textit{Firma}, by měl být zmíněným produktem výsledek této práce.

\textit{Firma}, jejíž zákazníci jsou z většiny výrobní společnosti, se zabývá optimalizací výroby a~její hlavní produkty jsou postaveny kolem tohoto problému, proto je pro ni důležité mít přístup ke kvalitním zákaznickým datům.
Potenciální zákazníci již většinou využívají nějaký ERP systém\footnote{Plánování podnikových zdrojů (ve zkratce ERP z anglického enterprise resource planning) je označení systému, jímž podnik (nebo jiná organizace) za pomoci počítače řídí a integruje všechny nebo většinu oblastí své činnosti, jako jsou plánování, zásoby, nákup, prodej, marketing, finance, personalistika, atd. \cite{ERP}.},
tudíž jsou i~jejich vnitřní procesy zmapované a~uložené v~jejich databázi.

Zpřístupnění dat zákazníků bylo původně řešeno vytvářením importovacích skriptů na míru zákazníkovi, což bylo časově náročné a~obtížně škálovatelné.
Dnes je pro nové zákazníky zadefinována sada databázových pohledů, které si musí sami naimplementovat a poskytnout \textit{Firmě}, která na ně napojí nabízené nástroje pro plánování a~optimalizaci výroby.
Tento proces v~reálných situacích mnohdy ústí ve dva problematické scénáře:

\begin{enumerate}
    \item {
        Nový zákazník nemá odborné znalosti potřebné k~vytvoření zmíněných databázových pohledů, proto osloví dodavatele jejich ERP systému s~žádostí o~nacenění potřebné úpravy.
        Odhadovaná cena za úpravy dodavatelem je příliš vysoká a zákazník o~\textit{Firmou} nabízený produkt ztrácí zájem.
    }
    \item {
        Nový zákazník databázové pohledy naimplementuje, ale v~datech má velké množství chyb, ať už z~důvodu chybně nastavených vnitřních procesů či chybné implementace pohledů.
        Tyto chyby jsou v~rozporu s~výše uvedeným požadavkem na kvalitní zákaznická data, a~tím pádem musí být nejprve odstraněny, což vytváří zdržení v~nasazování produktů \textit{Firmy}.
    }
\end{enumerate}

Pro omezení negativních důsledků výše zmíněných scénářů se firma rozhodla vytvořit nový produkt, který by usnadnil vytváření potřebné sady databázových pohledů pro import dat, umožnil vizualizaci takto importovaných dat a zároveň poskytl zákazníkovi okamžitou přidanou hodnotu.

Těmto požadavkům nejlépe odpovídala kombinace nástrojů pro mapování dat a~BI\footnote{
Souhrné označení pro dovednosti, znalosti, technologie, aplikace, a~postupy používané v~podnikání pro získání lepšího pochopení chování trhu a~obchodních souvislostí.
Za tímto účelem provádí sběr, integraci, analýzu, interpretaci a prezentaci obchodních informací.
}.
Odtud vychází zadání této bakalářské práce, jejíž cílem je, dle požadavků \textit{Firmy}, navržení a~implementace systému, který umožní poskytování předkonfigurovaného webového BI nástroje \textit{Metabase}\footnote{
Open-source BI nástroj umožňující jednoduchou vizualizaci dat, vytváření nástěnek a reportů. Více informací na \url{https://www.metabase.com/docs/latest/}
} 
formou SaaS (Software jako služba, anglicky Software as a~Service). 

Nástroj bude používat generický datový model vhodný pro většinu zákazníků, což umožní vytvoření základních reportů, které zrychlý adaptaci nástroje zákazníkem. Systém zákazníkům poskytne uživatelsky přívětivý nástroj pro mapování dat z~databáze jejich ERP systému na generický datový model a~přístup k~BI nástroji.
Správci systému umožní správu zákazníků, kontrolu mapování jejich dat a~snadný deploy instancí nástroje v Kubernetes klastru\footnote{Více informací v podsekci \ref{subsec:k8s} nebo na \url{https://kubernetes.io/docs/concepts/architecture/}.}. 

\section{Struktura práce}

Abychom popsali design, funkcionalitu a vývoj námi vytvářeného systému, rozdělíme dále práci na následujících 7 kapitol:

\subsubsection{Kapitola \ref{chap:analysis}: \nameref{chap:analysis}}
Zde si zasadíme námi vyvíjeny systém do kontextu již existujícího software.

\subsubsection{Kapitola \ref{chap:requirements}: \nameref{chap:requirements}}
V této kapitole si shrneme funkční a~nefunkční požadavky na tento systém.

\subsubsection{Kapitola \ref{chap:design}: \nameref{chap:design}}
Kapitola \ref{chap:design} popisuje, jak probíhal design tohoto systému, jaké rozhodnutí ho ovlivnili a jak reflektuje požadavky, které jsou na systém kladené.
Představujeme architekturu systému a~jednotlivých modulů, a~konečně představujeme zohlednění aspektů jako bezpečnost, použitelnost, škálovatelnost a~udržovatelnost systému.

\subsubsection{Kapitola \ref{chap:functionality}: \nameref{chap:functionality}}
V této kapitole ukazujeme funkcionality vyvíjeného systému, a~jak ji mohou zainteresované strany využít

\subsubsection{Kapitola \ref{chap:implementation}: \nameref{chap:implementation}}
Tato kapitola nastiňuje, jak jsme implementovali systém podle navrženého designu a požadavků.
Představujeme zde použité technologie, nástroje a~postupy, kterých jsme využili při vývoji systému.
Také popisujeme, jak jsme řešili některé problémy a~výzvy, na které jsme narazili během implementace.

\subsubsection{Kapitola \ref{chap:testing}: \nameref{chap:testing}}
Zde prezentujeme způsob testování funkčnosti, spolehlivosti a~kvality systému.

\subsubsection{Poslední kapitola: \nameref{chap:sum}}
V poslední části shrnujeme hlavní přínosy a~příspěvky naší práce. Zkoumáme, jak jsme splnili cíle a~požadavky na systém, a~jak jsme překonali některá omezení a~nedostatky.
Také diskutujeme o~možných budoucích rozšířeních a~vylepšeních systému, a~navrhujeme směry pro další výzkum.